\thispagestyle{empty}

\recipient{California NanoSystems Institute}{570 Westwood Plaza\\Building 114 | Mail Code: 722710\\Los Angeles, CA 722710}
\date{November 30, 2018}
\opening{Dear Dr. Rita Blaik,}
\closing{Sincerely,}
\enclosure[Attached]{curriculum vit\ae{} (online), Statement of Teaching, Service, and Contributions to Diversity (online)}          
\makelettertitle

Science and technology are increasingly a part of everyone's day-to-day life. From the GMO food that we eat to the computers we keep in our pocket, discussions about what technology should or shouldn't do and what scientists should or shouldn't prioritize are common. Experts provide valuable input for these discussions, but non-experts can and should have a seat at the table too. Effective and engaging science communication is key to giving non-experts the vocabulary they need to have their voices heard. Science communication does many vital things: inspires new generations of scientists, makes sure science is funded, and is entertaining (for some, at least). But underlying these goals is it's ability democratize science. By allowing dialogue between decision-makers, scientists \textit{and} non-scientists, more viewpoints are considered and ultimately better resolutions to conflict can be found. I plan to make this the focus of my science communication career and am excited to apply to the postdoctoral scholar position at the California NanoSystems Institute. 

Throughout my technical training, I have sought out experiences to develop and practice my communication skills in various mediums to diverse audiences. One experience in particular stands out. The Researchers Expanding Lay-Audience Teaching and Engagement (RELATE) program was my first formal training in communicating to a broader audience. I was excited to participate and learn the methods other scientists used to communicate their work. RELATE lasted three months, and during this time I wrote and produced a YouTube video about my research and gave a twenty minute talk to nearly 100 people at a local bar. The enthusiasim for my talk was encouraging, and I spent the rest of my night answering questions about light bulbs! Given just a little background, many people with minimal formal education in science were able to ask insightful questions about my research and were enthusiastic to do so. 

Throughout my undergraduate degree, I worked in various teaching assistant positions helping in both a laboratory setting and doing weekly recitation-like office hours. But in graduate school, I did not have many teaching opportunities because of my program and requirements on my funding. Nevertheless, rather than not teach at all, I found non-traditional ways of teaching. The American Society for Engineering Education provided the chance to teach smaller scale workshops on technical programs like Mathematica. We also developed small scale curricula, so that the weekly workshops would cover necessary skills through different instructors over multiple semesters. It was here I got exposed to discussions on newer teaching pedagogy and this lead to taking the Teaching Engineering course, where many of those discussions continued. Finally, I was lucky that my advisor frequently let me teach topics at group meeting and encouraged me to try different methods like active learning with my labmates.

Like other STEM PhDs, doing research has also given me many technical skills. Two of which are relevant to the position at CNSI. First, as a computational researcher, I am used to collaborating with other scientists outside my field. Working at CNSI, I will work with a wide array of scientists including biologists, doctors and various branches of engineering not in my specialty. This previous experience during my PhD will be invaluable as I talk to these experts and work on understanding their research and how to translate it to others. Second, a part of my PhD studies has been on nanotechnology. This will also be helpul in understanding the fundamental concepts of the research done at CNSI and translating them to different audiences.


While I have found the research done during my PhD exciting and interesting, the best parts of this experience have been the parts where I am telling someone else about my work. This postdoc at CNSI would allow me to take what I've been doing in my spare time and make it my full-time job. I would have more opportunities to learn about pedagogy during classes and more chances to put what I've learned into practice while working with the existing CNSI programming. I would be excited to work on developing new projects there, and would love to expand and think more about how CNSI can serve to connect scientists, policy makers and the public. My PhD has lead me to see this as a problem that my unique skills and interests can work towards. Through effective science communicaiton and engagement, everyone can learn about the exciting world of technology and have a say in how it develops.
\thispagestyle{empty}

I am very much looking forward to discussing my skills and experiences with you, and can be reached at mcala@umich.edu or at 732-275-5051.

\makeletterclosing
