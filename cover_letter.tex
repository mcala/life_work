\thispagestyle{empty}

%% recipient data
\recipient{University of Pennsylvania}{President's Center \\ Office of University Communications \\ 3901 Walnut Street \\ Philadelphia, PA 19104}
\date{August 9, 2018}
\opening{Dear hiring team,}
\closing{Sincerely,}
%\enclosure[Attached]{curriculum vit\ae{}}          % use an optional argument to use a string other than "Enclosure", or redefine \enclname
\makelettertitle

Communicating the importance of research being done at the University of Pennsylvania is vital for accomplishing the University's missions of inclusion, innovation, and excellence. By making the discoveries at Penn known in an engaging and understandable way, the University attracts a diverse body of world-class researchers and students who can continue to push at the boundaries of what our knowledge. Not only this, but this kind of communication inspires a new generation of scientists and allows non-experts to engage with research that affects their lives. I am excited to bring my skills and experiences to Penn to help with these challenges as a science news officer (position \#81-29227). I gained communicative, technical and social skills during my PhD that are applicable to this position.

Throughout my scientific training, I have sought out experiences to develop, practice and teach communication skills in various mediums and to various audiences. The RELATE (Researchers Expanding Lay-Audience Teaching and Engagement) program at University of Michigan was on major workshop where I worked on crafting messages and narratives, considering different audiences and making visual aids. Ultimately these experiences lead to creating a YouTube video and a talk about my research at a local bar. I also helped found the Students of Applied Physics program, where I am the senior editor. In this program, I work with other PhD students to write accessible articles about the research being done in the department. The experience in writing varies greatly among the students and I have learned a lot from teaching both big picture (storytelling, flow, style) and small scale (avoiding jargon, conciseness, proper grammar) topics in science writing.

Compared to most science PhD students I have spent a lot of time learning about communication topics. But I have also done original scientific research to get my PhD. While the specifics of my research will not be useful in this position, a vital part of my training has been learning how to quickly find, read, and see the big picture behind modern research. This skill is valuable as a science news officer as I work to report and understand scientific results quickly in a way that engages a diverse audience. This involves everything from being able to read a scientific paper quickly, managing citations, finding original sources, and distilling many papers on a single topic into something that can be shared with others in an understandable way. 

As a computational researcher, I frequently collaborate with other researchers from different fields like chemistry, mathematics, and materials scientists. Although my PhD expertise is not in these fields, I have learned how to communicate with other scientists -- both to understand their work and to explain my own. This ability will be valuable in the science news officer position, since I will be responsible for talking to faculty and deans in the departments of physics \& astronomy, mathematics, chemistry and the School of Engineering and Applied Science. I am confident that my technical background, combined with my ability to probe for big picture understanding, will allow me to write about the exciting projects being done across these departments at Penn.

Over the past year, in my spare time, I have published my writing on science for a general audience at the University of Michigan, Harvard’s Science in the News Blog, and on my personal website. Given the opportunity to work as a science news officer would allow me to put my excitement about writing and communicating first, rather than as a side interest. I am excited to discuss my skills and experiences with you and can be reached at mcala@umich.edu or at 732-275-5051.

\makeletterclosing
