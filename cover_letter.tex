\thispagestyle{empty}
\newgeometry{top=0.75in,bottom=0.75in,left=.9in,right=.9in}

%-----       letter       ---------------------------------------------------------

%% recipient data
\recipient{National Security Institute}{Antonin Scalia Law School \\ George Mason University \\ 3301 Fairfax Dr. \\ Arlington, VA 22201}
\date{December 7th, 2018}
\opening{Dear Fellowship Committee,}
\closing{Sincerely,}
%\enclosure[Attached]{Resume (online)}          % use an optional argument to use a string other than "Enclosure", or redefine \enclname
\makelettertitle

Science can provide solutions to both national and global scale challenges, but often those solutions come with large societal change. Making sure those changes are anticipated by the government is a part of a safe, stable and well functioning country. The National Security Institute helps provide policy makers with rigorous and objective understandings of new scientific developments in the domain of national security. Doing this requires understanding science and technology and being able to communicate it to non-experts. Both of these are skills that I have been working on throughout my PhD. But the third element, having a deep understanding of policy, is where I am lacking. This is why I am excited to apply for the NSI's Technologist Fellowship. This fellowship would help me better learn how to apply my technical and communicative skills towards helping policy-makers with problems in science and technology.

My technical experiences comes from my PhD, which I will be defending in January. This has given me many skills related to physics, materials science, and high-performance computing. Some of this has been from coursework, but much of it has been from searching and reading relevant technical literature. This ability to quickly find, read, and see the big picture behind modern research is a key part of translating between scientists and policymakers. I also have experience collaborating with other scientists. My work is entirely computational and I have worked closely with scientists in different fields. These collaborations have given me practice in understanding different methods of doing research and communicating between them. This is highly relevant to a job in public policy, where I would need to learn about an array of different subjects while working with different experts. While my own expertise is on the efficiency of light-emitting diode (LED) materials, I am confident I have the skills to work on a diverse set of scientific topics related to the hard questions facing the country.

Throughout my scientific training, I have also sought out experiences to develop, practice and teach communication skills in various mediums and to various audiences. I helped found the Students of Applied Physics program, where I am the senior editor. In this program, I work with other PhD students to write accessible articles about the research being done in our department. The writing experience varies greatly among the students and I have been able to develop my own skills by working with them. This involves teaching both big picture (storytelling, flow, style) and small scale (avoiding jargon, conciseness, proper grammar) topics in science communication. While these articles are for a different audience than I would be writing for in policy, the ideas behind effective writing and communicating can be applied to many different domains.

Unfortunately, I realized too late that public policy could connect my interest in science communication, science, and serving others. I didn't get to take advantage of all of the programs University of Michigan offers in science policy, but I did what I could with the time I had. Last winter, I took a course on science and technology policy. In addition to learning about science and technology policy around the world I had to practice distilling technical problems into succinct written documents appropriate for government audiences. More recently, I attended the National Science Policy Network's Annual Science Policy Symposium, where I got to learn about opportunities in policy, see examples of science policy in action, and network with like-minded graduate students.

After my defense I will be seeking a career in science policy. I hope to use my skills, experience and interests to serve the country and it's citizens. I will be applying to the Congressional Fellowships for scientific societies in the coming months and the NSI fellowship will better prepare me for the work I would be doing either as a fellow or in another related job. I'd be happy to talk more about my experiences, qualifications and career goals, and can be contacted at 732-275-5051 or mcala@umich.edu.

\makeletterclosing
\restoregeometry
