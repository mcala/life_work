\thispagestyle{empty}

%% recipient data
\recipient{University of Pennsylvania}{President's Center \\ Office of University Communications \\ 3901 Walnut Street \\ Philadelphia, PA 19104}
\date{August 16, 2018}
\opening{Dear Hiring Team,}
\closing{Sincerely,}
%\enclosure[Attached]{curriculum vit\ae{}}          % use an optional argument to use a string other than "Enclosure", or redefine \enclname
\makelettertitle

Communicating the importance of research being done at the University of Pennsylvania is vital for accomplishing the University's missions of inclusion, innovation, and excellence. Making the discoveries at Penn known in an engaging way attracts a diverse body of world-class researchers and students who can continue to push at the boundaries of our knowledge about the world. In addition, this communication also helps non-experts engage with research that often impacts their daily lives. I am excited to bring my skills and experiences to Penn to help with these challenges as a science news officer (position \#81-29227). During my PhD, I gained communicative, technical and social skills that are applicable to this position.

Throughout my scientific training, I have sought out experiences to develop, practice and teach communication skills in various mediums and to various audiences.  I helped found the Students of Applied Physics program, where I am the senior editor. In this program, I work with other PhD students to write accessible articles about the research being done in the department. This project serves two purposes. First, it helps prospective and new students see what research is being done in a very large department. Second, it gives current graduate students experience in writing for a general audience, a skill that is not emphasized in graduate school. Through teaching both big picture ideas (storytelling, flow, style) and small scale (avoiding jargon, conciseness, proper grammar) topics in science writing, I have improved my own understanding of what makes for an interesting story and how to tell it.

Compared to most science PhD students I have spent a large amount of time learning about communication topics. But I have also done original scientific research to get my PhD. While the specifics of my research will not be useful in this position, a part of my training has been learning how to quickly find, read, and see the big picture behind modern research. This skill is valuable as a science news officer as I work to understand and report scientific results in a way that engages a diverse audience. This involves knowing how to read a scientific paper, manage citations, find original sources, and distill many papers on a single topic into something that can be shared with others. These skills will be indispensable in the news officer position.

Finally, as a computational researcher, I frequently collaborate with other researchers from different fields like chemistry, mathematics, and materials science. These experiences have taught me how to communicate with other scientists -- both to understand their work and to explain my own.  This ability will be valuable in the science news officer position, since I will be responsible for talking to faculty and deans in the Departments of Physics \& Astronomy, Mathematics, Chemistry and the School of Engineering and Applied Science. I am confident that my technical background and my ability to probe for big picture understanding will lead to success writing about the exciting projects being done across these departments at Penn. I am excited to discuss my qualifications with you and can be reached at mcala@umich.edu or at 732-275-5051.

\makeletterclosing
