\thispagestyle{empty}

%-----       letter       ---------------------------------------------------------

%% recipient data
\recipient{IDA Science and Technology Policy Institute}{4850 Mark Center Drive \\ Alexandria, VA 22311}
\date{August 10, 2018}
\opening{Dear Dr. Ian Simon and the hiring team,}
\closing{Sincerely,}
%\enclosure[Attached]{curriculum vit\ae{}}          % use an optional argument to use a string other than "Enclosure", or redefine \enclname
\makelettertitle

Science can provide solutions to both national and global scale challenges, but often those solutions come with large societal change. Making sure those changes are anticipated by the government is a part of a stable and well functioning country. The Science and Technology Policy Institute helps provide rigorous and objective understanding of new scientific developments to government agencies. I am excited to bring my unique combination of skills and experience to STPI and support STPI's applied physics portfolio as a science and technology research analyst. My experiences throughout my doctorate have given me technical, social, and communicative abilities that would be valuable for this position.

My research experiences have given me many technical skills related to physics and materials science. Some of this has been from coursework, but much of it has been from searching and reading relevant technical literature. This ability to quickly find, read, and see the big picture behind modern research is a key part of the responsibilities of this position. I also have experience collaborating with other scientists. Because my work is entirely computational, I have worked closely with scientists in different fields. These collaborations have given me practice in understanding different methods and communicating between different research. This is highly relevant to a job at the STPI, where I would need to learn about an array of different subjects while working with different experts. While my own expertise is on efficiency and light-emitting diode (LED) materials, I am confident I have the skills to work on a diverse set of scientific topics, including space weather and other hazards that can affect our technological and infrastructure systems.

Throughout my scientific training, I have sought out experiences to develop, practice and teach communication skills in various mediums and to various audiences. I helped found the Students of Applied Physics program, where I am the senior editor. In this program, I work with other PhD students to write accessible articles about the research being done in the department. The writing experience varies greatly among the students and I have been able to develop my own skills by working with them. This involves teaching both big picture (storytelling, flow, style) and small scale (avoiding jargon, conciseness, proper grammar) topics in science communication. While these articles are for a different audience than I would be writing for at STPI, the ideas behind effective writing and communicating can be applied to different domains. I have already had a small change to do this in the policy domain, when I took a course on science and technology policy during the winter 2018 semester. In addition to learning about science and technology policy around the world (including some history of the OSTP, NSF and other organizations that STPI works with) I had to practice distilling technical problems into succinct written documents appropriate for government audiences. This knowledge and practice are directly relevant to a position at STPI.

I am looking forward to discussing my skills and experiences with you and can be reached at mcala@umich.edu or at 732-275-5051.

\makeletterclosing
