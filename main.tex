% possible options include font size ('10pt', '11pt' and '12pt'), paper size ('a4paper', 'letterpaper', 'a5paper', 'legalpaper', 'executivepaper' and 'landscape') and font family ('sans' and 'roman')
\documentclass[10pt,letter,sans,final,colorlinks,linkcolor=true]{moderncv}
\moderncvcolor{blue}
\moderncvstyle[left]{classic}

\usepackage{fontawesome}
\usepackage{multirow, multicol}

% to set the default font; use '\sfdefault' for the default sans serif font, '\rmdefault' for the default roman one, or any tex font name
%\renewcommand{\familydefault}{\sfdefault}

% uncomment to suppress automatic page numbering for CVs longer than one page
%\nopagenumbers{}

% character encoding
\usepackage[utf8]{inputenc}                       % if you are not using xelatex ou lualatex, replace by the encoding you are using

% adjust the page margins
\usepackage[scale=0.75]{geometry}
%\setlength{\hintscolumnwidth}{3cm}                % if you want to change the width of the column with the dates
%\setlength{\makecvheadnamewidth}{10cm}            % for the 'classic' style, if you want to force the width allocated to your name and avoid line breaks. be careful though, the length is normally calculated to avoid any overlap with your personal info; use this at your own typographical risks...


% personal data
\name{Andrew}{McAllister}
\title{PhD in Applied Physics, science communicator}
\address{3559 Burbank Drive}{Ann Arbor, MI 48105}{}% optional, remove / comment the line if not wanted; the "postcode city" and "country" arguments can be omitted or provided empty
\phone[mobile]{732-275-5051}                   % optional, remove / comment the line if not wanted; the optional "type" of the phone can be "mobile" (default), "fixed" or "fax"
\email{mcala@umich.edu}                               % optional, remove / comment the line if not wanted
\homepage{www.mcallister.science}                         % optional, remove / comment the line if not wanted
\social[twitter]{McAllisterSci}                             % optional, remove / comment the line if not wanted
\social[linkedin]{McAllisterSci}                        % optional, remove / comment the line if not wanted
%\social[github]{mcala}                              % optional, remove / comment the line if not wanted
%\extrainfo{additional information}                 % optional, remove / comment the line if not wanted
%\photo[64pt][0.4pt]{picture}                       % optional, remove / comment the line if not wanted; '64pt' is the height the picture must be resized to, 0.4pt is the thickness of the frame around it (put it to 0pt for no frame) and 'picture' is the name of the picture file
%\quote{Some quote}                                 % optional, remove / comment the line if not wanted

% bibliography adjustements (only useful if you make citations in your resume, or print a list of publications using BibTeX)
%   to show numerical labels in the bibliography (default is to show no labels)
%\makeatletter\renewcommand*{\bibliographyitemlabel}{\@biblabel{\arabic{enumiv}}}\makeatother
%\renewcommand*{\bibliographyitemlabel}{[\arabic{enumiv}]}
%   to redefine the bibliography heading string ("Publications")
%\renewcommand{\refname}{Articles}

% bibliography with mutiple entries
%\usepackage{multibib}
%\newcites{book,misc}{{Books},{Others}}
%----------------------------------------------------------------------------------
%            content
%----------------------------------------------------------------------------------
\begin{document}
\definecolor{links}{HTML}{0000FF}
\hypersetup{urlcolor=links}
%\begin{CJK*}{UTF8}{gbsn}                          % to typeset your resume in Chinese using CJK
%-----       resume       ---------------------------------------------------------
\vspace{-10cm}
\makecvtitle
\vspace{-1.3cm}

\section{Summary}
\cvline{Goal}{A career where I can use my technical expertise to
work on understanding complicated problems facing society and then communicate
those efforts (and possible solutions) to a wide variety of audiences.}
\cvline{Analytical Thinking}{A PhD in applied physics with specific expertise in materials science,
nanotechnology, energy efficiency, and high performance computing.}
\cvline{Communication}{Sought out specific training and experiences presenting to, writing for and
working with diverse audiences throughout my PhD.}

%----------------------------------------------------------------------------------------
% EDUCATION
% GPA isn't included. At this point, if they really care they'll ask for a transcript anyway.
%------------------------------------------------------------------------------------------

\section{Education}

\cventry{Expected: January 2019}{PhD in Applied Physics}{University of Michigan}{Ann
Arbor, MI}{}{\raggedright Relevant Coursework: \linebreak 
\vspace{-.3cm}
\begin{itemize}
\item Public Policy 650 -- Introduction to Science and Technology Policy Analysis 
\end{itemize}}
\cventry{2012}{B.S. in Physics}{Rensselaer Polytechnic Institute}{Troy, NY}{}
{Magna cum laude, dual major in mathematics}


%---------------------------------------------------------------------------------------- WORK
% WORK EXPERIENCE
%----------------------------------------------------------------------------------------

\section{Work Experience}

\cventry{2012-Present}{Graduate Student Researcher}{University of Michigan}{Ann Arbor, MI}{}{}

\cventry{June-August 2013}{Computational Chemistry and Materials Science
  Fellow}{\newline Lawrence Livermore National Laboratory}{Livermore, CA}{}{}

\cventry{2011-2012}{Undergraduate Research Assistant}{Rensselaer Polytechnic Institute}{\newline Troy, NY}{}{}


%----------------------------------------------------------------------------------------
% AWARDS
% Fellowships/Funding will have some overlap here.
%----------------------------------------------------------------------------------------

\section{Awards}
\cvitem{2014}{National Science Foundation Graduate Research Fellowship Program}
\cvitem{2012}{Nadia Trinkala Service Award, RPI Physics Department}
\cvitem{2011}{Pi Mu Epsilon, Mathematics Honor Society}
\cvitem{2010}{Rensselaer Polytechnic Institute Founder's Award of Excellence}
\cvitem{2008}{Boy Scouts of America, Eagle Scout}


%%----------------------------------------------------------------------------------------
% SKILLS
%----------------------------------------------------------------------------------------

\section{Programming Skills}
\cvitem{}{Languages: Fortran, Python, C++, Matlab, Shell, Git}
\cvitem{}{Materials Science Codes: QuantumEspresso, Wannier90, BerkeleyGW, VASP}
\cvitem{}{Further details and proficiencies available on request.}

\section{Leadership}
\cventry{2018-Present}{Organizer}{ComSciCon Michigan}{Ann Arbor, MI}{}{Work with
other graduate students to organize, publicize and run a conference devoted to
science communication in Ann Arbor Michigan.}
\cventry{2017-Present}{Senior Editor}{Students of Applied Physics
Project, Applied Physics Student Council}{Ann Arbor, MI}{}{Develop story ideas and edit articles that
PhD students write about each other's research.
\href{https://lsa.umich.edu/appliedphysics/news-events/all-news/ramon-martinez--measuring-chemical-fingerprints-with-super-conti.html}{Example
article}}
\cventry{2014-2015}{President}{Local Chapter of American Society for Engineering
Education}{Ann Arbor, MI}{}{Organize and run meetings, ensure that skill workshops
have teachers, plan future workshops based on the needs of University of
Michigan students.}
\cventry{2009-2011}{President}{Local Chapter of Society of Physics
Students}{Troy, NY}{}{Organize meetings and social events, foster a community of
physics students, act as intermediary between faculty and students, help
organize and run engagement events in local area.}

%\cventry{2013-2016}{American Society for Engineering Education}{}{}{}{Organized
%and ran a table at K-Grams Kid's Fair --  an elementary school visit to University
%of Michigan. At the table, I helped demonstrate some concepts of signal analysis by using a
%laser to transmit music through open air.}
%\cvlistitem{President of local organization: 2014-2015}
%\cventry{2008-2012}{Society of Physics Students}{}{}{}{Organized
%and ran multiple outreach events at local schools and on campus. A large project
%that I was involved with was organizing a full-day program on light and solar cells
%for the Harlem Academy's visit to Rensselaer with my advisor, Peter Persans.}
%\cvlistitem{President of local organization: 2009-2011}


\section{Communication Training}
\cventry{August 2017}{\href{https://comscicon.com/comscicon-chicago-2017}{ComSciCon
Chicago [Link for more information]}}{Chicago, IL}{}{}{Attended a conference based on
science communication.}
\cventry{2016}{Researchers Expanding Lay-Audience Teaching and Engagement
(RELATE) Workshops}{Ann Arbor, MI}{}{}{}
\cvlistitem{Over 3 months, worked on crafting messages and narratives, considering different audiences and making visual aids.}
\cvlistitem{Developed and produced a \href{https://www.youtube.com/watch?v=2IxPJ6lhGhA}{YouTube video [Link]} highlighting
my research.}


\section{Selected Communication Experience}

\vspace{.15cm}
\subsection{General Audience Writing}
\cventry{1. }{\href{http://www.mse.engin.umich.edu/about/news/atomistic-calculations-predict-that-boron-incorporation-increases-the-efficiency-of-leds-6}{Atomistic
Calculations Predict That Boron Incorporation Increases The Efficiency Of
LEDs}}{2017}{}{}{Press release for research group. Picked up by the DOE, NERSC,
and Semiconductor Today.}

\cventry{2. }{Senior Editor}{Students of Applied Physics
, Applied Physics Student Council}{}{}{
I work with PhD students to develop understandable and engaging articles about
research in the applied physics department.  \href{https://lsa.umich.edu/appliedphysics/news-events/all-news/ramon-martinez--measuring-chemical-fingerprints-with-super-conti.html}{Example
article}}
%----------------------------------------------------------------------------------------
% PRESENTATIONS
% If you ever get an invited talk to a conference: invited conference presentations
% If you ever get an invited talk to somewhere: invited seminar
%----------------------------------------------------------------------------------------

\vspace{.25cm}
\subsection{Public Engagement}

\cvitem{1.}{\textbf{Andrew McAllister},
\href{https://aadl.org/node/353283}{LED Light Bulbs: Why Do They Cost an Arm and
a Leg?}, \href{https://annarbor.nerdnite.com}{Nerd Nite} 2017, Ann Arbor, MI}

%----------------------------------------------------------------------------------------
% PUBLICATIONS
% Make sure to bold your name, have the doi link, and have the same formatting for all papers.
% Eventually you may want an easier way to add papers than by hand, (using bibtex) but for now this
% will do.
% Book Chapters (if there ever are any, will be a subsection here)
%----------------------------------------------------------------------------------------
\vspace{.25cm}
\subsection{Technical Publications}

\cvitem{1.}{\textbf{Andrew McAllister}, Dylan Bayerl, Emmanouil Kioupakis, Auger
and radiative recombination in indium nitride, \textit{Applied Physics Letters},
\textbf{112}, 251108 (2018)
\href{https://doi.org/10.1063/1.5038106}{doi:10.1063/1.5038106}}

\cvitem{2.}{Kyeongwoon Chung, \textbf{Andrew McAllister}, David Bilby, Bong-Gi Kim, Min Sang Kwon,
  Emmanouil Kioupakis, Jinsang Kim, Designing interchain and intrachain properties of conjugated
  polymers for latent optical information encoding, \textit{Chemical Science} \textbf{6}, 6980-6985
  (2015) \href{http://dx.doi.org/10.1039/C5SC02403J}{doi:10.1039/c5sc02403j}}

\vspace{.25cm}
\subsection{Contributed Technical Presentations}

\cvitem{1.}{\textbf{Andrew McAllister}, Dylan Bayerl, Christina Jones, Emmanouil Kioupakis, Auger Recombination From First-principles in Group-III Nitride Alloys, American Physical Society March Meeting 2018, Los Angeles, CA}

\cvitem{2.}{\textbf{Andrew McAllister}, Dylan Bayerl, Emmanouil Kioupakis, Radiative and Auger Recombination of Degenerate Carriers in InN American Physical Society March Meeting, 2017, New Orleans, LA}

\cvitem{3.}{\textbf{Andrew McAllister}, Predictive modeling of quantum
  processes for optoelectronic devices, Physics Graduate Student Symposium,
  2014, Ann Arbor, MI}


%%----------------------------------------------------------------------------------------
% PUBLICATIONS
% Make sure to bold your name, have the doi link, and have the same formatting for all papers.
% Eventually you may want an easier way to add papers than by hand, (using bibtex) but for now this
% will do.
% Book Chapters (if there ever are any, will be a subsection here)
%----------------------------------------------------------------------------------------
\section{Technical Publications}
\cvitem{1.}{Jimmy-Xuan Shen, Daniel Steiauf, \textbf{Andrew McAllister},
 Guangsha Shi, Emmanouil Kioupakis, Anderson Janotti, and Chris Van de Walle,
 Impact of phonons and spin-orbit coupling on Auger recombination in InAs,
 \textit{submitted}}

\cvitem{2.}{\textbf{Andrew McAllister}, Dylan Bayerl, Emmanouil Kioupakis, Auger
and radiative recombination in indium nitride, \textit{Applied Physics Letters},
\textbf{112}, 251108 (2018)
\href{https://doi.org/10.1063/1.5038106}{doi:10.1063/1.5038106}}

 \cvitem{3.}{Kyeongwoon Chung, \textbf{Andrew McAllister}, David Bilby, Bong-Gi Kim, Min Sang Kwon,
   Emmanouil Kioupakis, Jinsang Kim, Designing interchain and intrachain properties of conjugated
   polymers for latent optical information encoding, \textit{Chemical Science} \textbf{6}, 6980-6985
   (2015) \href{http://dx.doi.org/10.1039/C5SC02403J}{doi:10.1039/c5sc02403j}}

 \cvitem{4.}{\textbf{Andrew McAllister}, Daniel \r{A}berg, Andr\'e Schleife, and Emmanouil Kioupakis,
   Auger recombination in sodium-iodide scintillators from first principles, \textit{Applied Physics
     Letters} \textbf{106}, 141901 (2015) \href{http://dx.doi.org/10.1063/1.4914500}{doi:10.1063/1.4914500}}

 \cvitem{5.}{Daniel Recht, David Hutchinson, Thomas Cruson, Anthony DiFranzo, \textbf{Andrew
     McAllister}, Aurore J. Said, Jeffrey M. Warrender, Peter D. Persans, and Michael J. Aziz,
   Contactless Microwave Measurements of Photoconductivity in Silicon Hyperdoped with Chalcogens,
   \textit{Applied Physics Express} \textbf{5}, 041301 (2012)
   \href{http://dx.doi.org/10.1143/APEX.5.041301}{doi:10.1143/apex.5041301}}

%\section{Writing and Editing for a General Audience}

\vspace{.15cm}

\cventry{1.}{\href{http://sitn.hms.harvard.edu/flash/2018/how-to-talk-to-your-plants/}{Using
LEDs to Tell Plants What We Want From Them [Link]}}{Science in the
News Blog, 2018}{}{}{Worked with the "Friends of Joe's Big Idea" program by
National Public Radio.}

\cventry{2.}{Senior Editor}{Students of Applied Physics, Applied Physics Student Council}{}{}{
I work with PhD students to develop understandable and engaging articles about
research in the applied physics department.
\href{https://lsa.umich.edu/appliedphysics/news-events/all-news/ramon-martinez--measuring-chemical-fingerprints-with-super-conti.html}{Example
article [Link]}}

\cventry{3.}{\href{http://www.mse.engin.umich.edu/about/news/atomistic-calculations-predict-that-boron-incorporation-increases-the-efficiency-of-leds-6}{Atomistic
Calculations Predict That Boron Incorporation Increases The Efficiency Of LEDs
[Link]}}{University of Michigan Materials Science \& Engineering News,
2017}{}{}{Press release for research group. Picked up by the Department of
Energy, National Energy Research Scientific Computer Center, and Semiconductor Today.}

\cventry{4.}{\href{https://misciwriters.com/2017/06/06/2827/}{How Gecko Feet
Will Make Your Next Move Easier [Link]}}{Michigan Science Writers, 2017}{}{}{I also
work as a content editor for Michigan Science Writers, where I provide feedback
and help develop a rought draft developing of a piece by another graduate student.}

%%----------------------------------------------------------------------------------------
% PRESENTATIONS
% If you ever get an invited talk to a conference: invited conference presentations
% If you ever get an invited talk to somewhere: invited seminar
%----------------------------------------------------------------------------------------

\section{Selected Presentations}
\subsection{Contributed Technical Oral Presentations}

\cvitem{1.}{\textbf{Andrew McAllister}, Dylan Bayerl, Christina Jones, Emmanouil Kioupakis, Auger Recombination From First-principles in Group-III Nitride Alloys, American Physical Society March Meeting 2018, Los Angeles, CA}

\cvitem{2.}{\textbf{Andrew McAllister}, Dylan Bayerl, Emmanouil Kioupakis, Radiative and Auger Recombination of Degenerate Carriers in InN American Physical Society March Meeting, 2017, New Orleans, LA}

\cvitem{3.}{\textbf{Andrew McAllister}, Emmanouil Kioupakis, Daniel \r{A}berg, Andre\'e Schleife,
  Auger recombination in scintillator materials from first principles, American Physical Society
  March Meeting, 2015, San Antonio, TX}

\cvitem{4.}{\textbf{Andrew McAllister}, Predictive modeling of quantum
  processes for optoelectronic devices, Physics Graduate Student Symposium,
  2014, Ann Arbor, MI}

\subsection{Public Engagement}

\cventry{1.}{\href{https://annarbor.nerdnite.com}{Nerd Nite [Link]} Ann Arbor
Talk}{}{}{}{Gave a 20 minute talk about my research at
a local bar to an audience of mostly non-scientists. A recording is
available at: \href{https://aadl.org/node/353283}{LED Light Bulbs: Why Do They
Cost an Arm and a Leg? [Link]}}

%%----------------------------------------------------------------------------------------
% PUBLIC ENGAGEMENT 
%----------------------------------------------------------------------------------------

\section{Public Engagement}
\cventry{2018}{Engaging Scientists in Policy and Advocacy}{}{}{}{Voluteer for
"Ask a Scientist at Art Fair", where I spoke to adults interested in science at
a large local event in an informal setting.}
\cventry{2018}{\href{https://www.skypeascientist.com}{Skype a
Scientist [Link]}}{}{}{}{Volunteered for the Skype a Scientist program, where I skyped
into multiple high school classrooms to talk about science, becoming a
scientist, and other topics.  More information on my blog,
\href{https://www.mcallister.science/skype-a-scientist}{here. [Link]}}
\cventry{2017}{\href{https://annarbor.nerdnite.com}{Nerd Nite [Link]} Ann Arbor
Talk}{}{}{}{Gave a 20 minute talk about my research at
a local bar to an audience of mostly non-scientists. A recording of the talk is
available at: \href{https://aadl.org/node/353283}{LED Light Bulbs: Why Do They
Cost an Arm and a Leg? [Link]}}
\cventry{2013-2016}{American Society for Engineering Education}{}{}{}{Organized
and ran a table at K-Grams Kid's Fair --  an elementary school visit to University
of Michigan. At the table, I helped demonstrate some concepts of signal analysis by using a
laser to transmit music through open air.}
\cventry{2008-2012}{Society of Physics Students}{}{}{}{Organized
and ran multiple outreach events at local schools and on campus. A large project
that I was involved with was organizing a full-day program on light and solar cells
for the Harlem Academy's visit to Rensselaer with my advisor, Peter Persans.}


%
%----------------------------------------------------------------------------------------
% TEACHING
% This is pretty ugly, mostly since things spill onto multiple lines. But I'm not sure how to handle
% this so it will be future Andrew's problem. THANKS PAST ANDREW YOU SUCK. 
% Why did I think this was such a hard problem to solve, not once but twice? 2.28.2017
%----------------------------------------------------------------------------------------

% Add course?

\section{Teaching Experience}
\cvitem{}{\emph{At Concordia University Ann Arbor:}}
\cvitem{Spring 2019}{Physics 152 -- General Physics II}
\cvitem{}{\emph{At the University of Michigan:}}
\cvitem{April 2015}{Flow in Technical Writing Workshop}
\cvitem{October 2014}{Introduction to Mathematica Workshop}
\cvitem{April 2014}{Introduction to \LaTeX\  Workshop}
\vspace{0.25cm}
\cvitem{}{\emph{At Rensselaer Polytechnic Institute:}}
\cvitem{Spring 2012}{Teaching Assistant, Physics 4100 - Introductory Quantum Mechanics}
\cvitem{Fall 2011}{Teaching Assistant, Physics 2961 - Modern Physics}
\cvitem{Fall 2011}{Grader, Math 4400 - Ordinary Differential Equations}
\cvitem{Spring 2011}{Teaching Assistant, Physics 1200 - Introductory Electromagnetism}
\cvitem{Fall 2010}{Teaching Assistant, Physics 1200 - Introductory Electromagnetism}

%%----------------------------------------------------------------------------------------
% Coursework
% 
%----------------------------------------------------------------------------------------
\section{Other Education Experiences}

\cventry{Winter 2018}{Public Policy 650 - Introduction to Science and Technology Policy
Analysis}{\newline University of Michigan}{}{}{}
\cventry{August 2017}{\href{https://comscicon.com/comscicon-chicago-2017}{ComSciCon
Chicago}}{Chicago, IL}{}{}{}
\cventry{Fall 2016}{Engineering 580 - Teaching Engineering}{University of Michigan}{}{}{}




%%----------------------------------------------------------------------------------------
% PROFESSIONAL MEMBERSHIPS
%----------------------------------------------------------------------------------------

\section{Professional Memberships}

%\cvitem{}{American Association for the Advancement of Science}
%\cvitem{}{American Physical Society}
%\cvitem{}{American Society for Engineering Education}
%\cvitem{}{Materials Research Society}
%\cvitem{}{Society for Social Studies of Science}

% One liner
\cvitem{}{American Association for the Advancement of Science\separator American Physical Society\separator American Society for Engineering Education\separator Materials Research Society\separator Society for the Social Studies of Science}

%%----------------------------------------------------------------------------------------
% GRANTS/FUNDING
% Right now you don't need a separate section for this. But the high performance computing rewards
% fall under this category, although in their own section.
%----------------------------------------------------------------------------------------

\section{High-Performance Computing Awards}

\cvitem{2015-2018}{Electronic and optical properties of novel photovoltaic and thermoelectric materials
  from first-principles, National Energy Research Scientific Computing Center \break 
  PI: Emmanouil Kioupakis}
  \cvlistitem{\textbf{2018:} 5,000,000 CPU Hours}
  \cvlistitem{\textbf{2017:} 7,300,000 CPU Hours}
  \cvlistitem{\textbf{2016:} 2,301,200 CPU Hours}
  \cvlistitem{\textbf{2015:} 8,000,000 CPU Hours}

%%----------------------------------------------------------------------------------------
% MENTORING
% Eventually can put undergrads who you help out with in the lab here. Also can add lunch + AP
% mentoring things.
% ADD NEXT TIME.
%----------------------------------------------------------------------------------------

\section{Mentoring}

\cvitem{2014}{Lunch and Lab with a Grad Mentoring Program}



% Publications from a BibTeX file without multibib
%  for numerical labels: \renewcommand{\bibliographyitemlabel}{\@biblabel{\arabic{enumiv}}}% CONSIDER MERGING WITH PREAMBLE PART
%  to redefine the heading string ("Publications"): \renewcommand{\refname}{Articles}
%\nocite{*}
%\bibliographystyle{plain}
%\bibliography{publications}                        % 'publications' is the name of a BibTeX file

% Publications from a BibTeX file using the multibib package
%\section{Publications}
%\nocitebook{book1,book2}
%\bibliographystylebook{plain}
%\bibliographybook{publications}                   % 'publications' is the name of a BibTeX file
%\nocitemisc{misc1,misc2,misc3}
%\bibliographystylemisc{plain}
%\bibliographymisc{publications}                   % 'publications' is the name of a BibTeX file

\clearpage
%-----       letter       ---------------------------------------------------------

%% recipient data
\recipient{Company Recruitment team}{Company, Inc.\\123 somestreet\\some city}
\date{June 22, 2018}
\opening{Dear Sir or Madam,}
\closing{Yours faithfully,}
\enclosure[Attached]{curriculum vit\ae{}}          % use an optional argument to use a string other than "Enclosure", or redefine \enclname
\makelettertitle

Lorem ipsum dolor sit amet, consectetur adipiscing elit. Duis ullamcorper neque sit amet lectus facilisis sed luctus nisl iaculis. Vivamus at neque arcu, sed tempor quam. Curabitur pharetra tincidunt tincidunt. Morbi volutpat feugiat mauris, quis tempor neque vehicula volutpat. Duis tristique justo vel massa fermentum accumsan. Mauris ante elit, feugiat vestibulum tempor eget, eleifend ac ipsum. Donec scelerisque lobortis ipsum eu vestibulum. Pellentesque vel massa at felis accumsan rhoncus.

Suspendisse commodo, massa eu congue tincidunt, elit mauris pellentesque orci, cursus tempor odio nisl euismod augue. Aliquam adipiscing nibh ut odio sodales et pulvinar tortor laoreet. Mauris a accumsan ligula. Class aptent taciti sociosqu ad litora torquent per conubia nostra, per inceptos himenaeos. Suspendisse vulputate sem vehicula ipsum varius nec tempus dui dapibus. Phasellus et est urna, ut auctor erat. Sed tincidunt odio id odio aliquam mattis. Donec sapien nulla, feugiat eget adipiscing sit amet, lacinia ut dolor. Phasellus tincidunt, leo a fringilla consectetur, felis diam aliquam urna, vitae aliquet lectus orci nec velit. Vivamus dapibus varius blandit.

Duis sit amet magna ante, at sodales diam. Aenean consectetur porta risus et sagittis. Ut interdum, enim varius pellentesque tincidunt, magna libero sodales tortor, ut fermentum nunc metus a ante. Vivamus odio leo, tincidunt eu luctus ut, sollicitudin sit amet metus. Nunc sed orci lectus. Ut sodales magna sed velit volutpat sit amet pulvinar diam venenatis.

Albert Einstein discovered that $e=mc^2$ in 1905.


\makeletterclosing

\end{document}
